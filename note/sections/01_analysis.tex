\section{Введение}
\label{sec:introduction}

\subsection{Актуальность и значение темы}
Человечество с давних времён пытается описать наш мир в виде формул и строгих определений. Многие процессы в мире можно описать некой функциональной зависимостью вида $y=f(x)$. Поэтому очень часто возникают задачи по нахождению либо фукнции $f(x)$, либо её аргументов, при которых функция принимает требуемое значение. Подвидом таких задач являются задачи на нахождение таких аргументов этой функции, при которых она бы обращалась в ноль, то есть задачи по решению уравнений вида
\begin{equation}
	\label{intro:formula_def}
	f(x)=0
\end{equation}

И хотя для удобства использования человечество пытается очень многие процессы свести к линейным зависимостям, для таких сфер как физика или экономика это может быть попросту невозможно из-за сложности или хаотичности поведения объектов в этих сферах. Поэтому для таких задач, которые используют такие уравнения, разрабатываются многие методы их решения, которые пытаются если не точно решить такие задачи (как, например, в случае с квадратными уравнениями вида $ax^2 + bx + c = 0$, формулы нахождения корней которых давно известны и достаточно просты), то как минимум найти решения этих уравнений с точностью, необходимой для конкретной задачи. И человечество до сих пор находит новые и более изощрённые методы решения подобных систем, иногда с использованием компьютерных технологий.

\subsection{Формулировка задачи и цели}
Из вышеописанного становится понятным значение этой темы для мирового научного сообщества. Исходя из этого, для этой курсовой работы была поставлена задача исследовать одни из наиболее популярных и тривиальных методов решения подобных систем и реализовать эти методы с использованием языка программирования \textit{C++}. 
Для анализа были выбраны метод простых итераций, ``метод Ньютона'' и метод секущих.

Конечной целью этой курсовой работы является реализованная и качественно задокументированная программа на языке \textit{C++}, реализующая каждый из представленных выше методов и предоставляющая пользователю возможность ввода произвольных входных данных для получения последующего ответа, содержащего корни уравнения или сообщения, что найти такие корни при выбранных исходных данных невозможно.

